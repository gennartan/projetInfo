\documentclass[10pt,a4paper]{article}
\usepackage[utf8]{inputenc}
\usepackage[francais]{babel}
\usepackage[T1]{fontenc}
\usepackage{amsmath}
\usepackage{amsfonts}
\usepackage{amssymb}
\usepackage[left=2.5cm,right=2.5cm,top=2.5cm,bottom=2.5cm]{geometry}
\author{Antoine Gennart}
\title{Rapport de projet \newline  Projet Informatique 2 : Qui Oz-ce ?}
\date{\today}
\begin{document}
\maketitle

\tableofcontents

\section*{Introduction}
Dans le cadre du cours d'informatique 2 de la faculté des siences appliquée de Louvain la Neuve, il nous a été demandé de réaliser le célèbre jeu "\textit{Qui est-ce}".

\section{Description de l'algorithme}
Nous devons implémenter deux fonctions principales : \textbf{BuildDecisionTree} qui va créer un arbre de décision le plus optimal possible pour que l'ordinateur pose le moins de questions possibles pour tomber sur la bonne personne. Et \textbf{GameDriver}, qui va, à partir de l'arbre créer par le BuildDecisionTree, décider des questions à poser, et s'il n'y a plus de questions a poser, proposer une solution.

\subsection{BuildDecisionTree}


\subsection{GameDriver}


\end{document}