\documentclass[10pt,a4paper]{article}
\usepackage[utf8]{inputenc}
\usepackage[francais]{babel}
\usepackage[T1]{fontenc}
\usepackage{amsmath}
\usepackage{amsfonts}
\usepackage{amssymb}
\usepackage[left=2.5cm,right=2.5cm,top=2.5cm,bottom=2.5cm]{geometry}
\author{Antoine Gennart}
\title{Eplication de l'algorithme}
\begin{document}
\maketitle

Ce document est destiné à décrire tout l'algoritme du code du fichier code.oz. Il n'est pas destiné à un quelconque proffesseur, mais juste à éclaircir la compréhension du code et des environnements contextuels des différentes fonctions et sous fonctions.

Pour rappel, l'objectif du code est de gérer un jeu de "Qui-est ce ?" dans le très répandu language informatique OZ

\tableofcontents

\section{BuildDecisionTreeWithQ}
\paragraph{Description :} Crée un arbre de décision à partir de du base de donnée de personnes et d'une liste de personnes.

\begin{itemize}
\item Gardener
\item BestQuestion
\end{itemize}

\subsection{MakeListOfNames}
\paragraph{Description :} A partir d'une liste de person(NOM Q:true ..... false), crée une liste contenant uniquement les NOM.

\begin{itemize}
\item MakeListAcc
\end{itemize}

\subsection{Gardener}
\paragraph{Description :} C'est réellement le jardinier du programme, c'est lui qui s'occupe de créer l'arbre de manière récursive.

\begin{itemize}
\item BestQuestion
\item MakeListOfNames
\end{itemize}

\section{BestQuestion}
\paragraph{Description :} Choisis dans une liste de questions laquelle conviendra le mieux pour éliminer le plus de personnes possible d'une certaine base de donnée.

\begin{itemize}
\item DiffTrueFalseList
\item MinList
\item RemoveList
\end{itemize}

\subsection{RemoveList}
\paragraph{Description :} Retire le Nième élément d'une liste

\begin{itemize}
\item RemoveListAcc
\end{itemize}

\subsection{MinList}
\paragraph{Description :} Retourne l'indice de l'emplacement du plus petit élément de la liste

\begin{itemize}
\item MinListACc
\end{itemize}

\subsection{DiffTrueFalseList}
\paragraph{Description :} Crée une liste dont chaque élément est la valeur absolue de la différence des réponses true et false par les personnes à laquelle on soustrait le nombre de personne qui ont répondu a la question.

\begin{itemize}
\item DiffTrueFalseListAcc
\end{itemize}

\section{BuildDecisionTree}
\paragraph{Description :} Crée un arbre de décision juste à partir de la base de donnée. C'est cette fonction que l'on doit donner  au player pour lancer l'interface graphique.

\begin{itemize}
\item MakeListOfQuestions
\item BuildDecisionTreeWithQ
\end{itemize}

\subsection{MakeListOfQuestions}
\paragraph{Description :} Fait une liste des questions a partir de toutes celle qui apparaisent dans une base de donnée.

\begin{itemize}
\item MakeListOfQuestionsAcc
\item RemoveDouble
\end{itemize}

\section{RemoveDouble}
\paragraph{Description :} Retire toutes les valeurs qui apparaissent deux fois dans une liste.

\begin{itemize}
\item RemoveDoubleAcc
\item IsIn
\end{itemize}

\subsection{IsIn}
\paragraph{Description :} Fonction Boolean qui regarde si un élément appartient à une liste

\section{GameDriveer}
\paragraph{Description :} Pilote du jeux qui va se ballader dans l'arbre pour poser les bonnes questions à l'utilisateur. Si l'utilisateur est con et qu'il ne sait pas, il doit recreer un arbre.

\begin{itemize}
\item GameDriverAcc
\end{itemize}

\subsection{NewDataBase}
\paragraph{Description :} Met à jour la base de donnée en retirant les personnes qui n'ont pas la même réponses que celle décidée par l'utilisateur.

\begin{itemize}
\item NewDBAcc
\end{itemize}

\subsection{NewListOfQuestions}
\paragraph{Description :} Retire la dernière questions posée de la liste des questions.

\begin{itemize}
\item RemoveList
\end{itemize}

\subsection{GameDriverAcc}
\paragraph{Description :} Fonction récursive du GameDriver.

\begin{itemize}
\item NewListOfQuestions
\item NewDataBase
\item BuildDecisionTreeWithQ
\item GameDriverAcc
\end{itemize}


\end{document}