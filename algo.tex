\documentclass[10pt,a4paper]{article}
\usepackage[utf8]{inputenc}
\usepackage[francais]{babel}
\usepackage[T1]{fontenc}
\usepackage{amsmath}
\usepackage{amsfonts}
\usepackage{amssymb}
\usepackage[left=2.5cm,right=2.5cm,top=2.5cm,bottom=2.5cm]{geometry}
\author{Antoine Gennart \& Marie-Pierre van Oldeneel}
\title{Explication de l'algorithme}
\begin{document}
\maketitle

Ce document est destiné à décrire tout l'algoritme du code du fichier code.oz. Il n'est pas destiné à un quelconque proffesseur, mais juste à éclaircir la compréhension du code et des environnements contextuels des différentes fonctions et sous fonctions.

Pour rappel, l'objectif du code est de gérer un jeu de "Qui-est ce ?" dans le très répandu language informatique OZ

\tableofcontents
\newpage

\section{BuildDecisionTreeWithQ}
\paragraph{Description :} Crée un arbre de décision à partir de du base de donnée de personnes et d'une liste de personnes.

\begin{itemize}
\item Gardener
\item BestQuestion
\end{itemize}

\subsection{MakeListOfNames L}
\paragraph{Description :} A partir d'une liste de person(NOM Q:true ..... false), crée une liste contenant uniquement les NOM.

\begin{itemize}
\item MakeListAcc
\end{itemize}

Complexité temporelle : $O(n)$ avec n la taille de liste L.

\subsection{Gardener DB ListT ListF ActualQuestion ListOfQuestions}
\paragraph{Description :} C'est réellement le jardinier du programme, c'est lui qui s'occupe de créer l'arbre de manière récursive.

\begin{itemize}
\item BestQuestion
\item MakeListOfNames
\end{itemize}

A chaque itération, Gardener se sépare en deux dimiuant toujours la taille de la DB.

Complexité temporelle : La fonction se rappelle a chaque itération deux fois (typiquement $\Theta(n^2)$, avec n la taille de la liste de questions), en changeant ses argument avec l'aide de la fonction BestQuestion ($O(n_1 \dot n_2)$).

On peut donc définir la complexité temporelle de cette fonction comme $\Theta(n_1 \cdot n_2^2)$ ou $n_2$ est la taille de la liste de question, $n_1$ la taille de la DB.

\section{BestQuestion DB ListOfQuestions}
\paragraph{Description :} Choisis dans une liste de questions laquelle conviendra le mieux pour éliminer le plus de personnes possible d'une certaine base de donnée.

\begin{itemize}
\item DiffTrueFalseList
\item MinList
\item RemoveList
\end{itemize}

Connaissant la complexitées temporelles des sous fonction, on peut calculer la complexité temporelle de BestQuestion qui appelle chacune des fonctions séparément.

Complexité Temporelle : $O(n_1 \cdot n_2) + O(n_2) + O(n_2)$ ou $n_1$ est la taille de DB et $n_2$ la taille de ListOfQuestions

\subsection{RemoveList L Nth}
\paragraph{Description :} Retire le Nième élément d'une liste

\begin{itemize}
\item RemoveListAcc
\end{itemize}

Complexité temporelle : $O(n)$ ou n est la longueur de la liste 

\subsection{MinList L}
\paragraph{Description :} Retourne l'indice de l'emplacement du plus petit élément de la liste

\begin{itemize}
\item MinListACc
\end{itemize}

Complexité temporelle : $O(n)$ ou n est la longueur de la liste L

\subsection{DiffTrueFalseList DB ListOfQuestions}
\paragraph{Description :} Crée une liste dont chaque élément est la valeur absolue de la différence des réponses true et false par les personnes à laquelle on soustrait le nombre de personne qui ont répondu a la question.

\begin{itemize}
\item DiffTrueFalseListAcc
\end{itemize}

Complexité temporelle : $O(n1 \cdot n2)$ ou n1 est la taille de la DB et n2 la taille de ListOfQuestions

\section{BuildDecisionTree}
\paragraph{Description :} Crée un arbre de décision juste à partir de la base de donnée. C'est cette fonction que l'on doit donner  au player pour lancer l'interface graphique.

\begin{itemize}
\item MakeListOfQuestions
\item BuildDecisionTreeWithQ
\end{itemize}

\section{MakeListOfQuestions DB}
\paragraph{Description :} Fait une liste des questions a partir de toutes celle qui apparaisent dans une base de donnée.

\begin{itemize}
\item MakeListOfQuestionsAcc
\item RemoveDouble
\end{itemize}

Complexité temporelle : $\Theta(n_1 \cdot n_2)$ avec $n_1$ la taille de la base de donnee et $n_2$ le nombre total de questions (de tous les joueurs).

\subsection{RemoveDouble List}
\paragraph{Description :} Retire toutes les valeurs qui apparaissent deux fois dans une liste.

\begin{itemize}
\item RemoveDoubleAcc
\item IsIn
\end{itemize}

Complexité temporelle : $O(n)$ ou n est la taille de la List

\subsection{IsIn X L}
\paragraph{Description :} Fonction Boolean qui regarde si un élément appartient à une liste

Complexité temporelle : $\Theta(n)$, $\Omega(1)$ ou n est la taille de la liste L

\section{GameDriver}
\paragraph{Description :} Pilote du jeux qui va se ballader dans l'arbre pour poser les bonnes questions à l'utilisateur. Si l'utilisateur est con et qu'il ne sait pas, il doit recreer un arbre.

\begin{itemize}
\item GameDriverAcc
\end{itemize}

\subsection{NewDataBase DB}
\paragraph{Description :} Met à jour la base de donnée en retirant les personnes qui n'ont pas la même réponses que celle décidée par l'utilisateur.

\begin{itemize}
\item NewDBAcc
\end{itemize}

Complexité temporelle : $O(n)$ ou $n$ est la taille de la DB

\subsection{NewListOfQuestions ActualQ LQ}
\paragraph{Description :} Retire la dernière questions posée de la liste des questions.

\begin{itemize}
\item RemoveList
\end{itemize}

Complexité temporelle : $O(n)$ ou $n$ est la taille de la liste de questions

\subsection{GameDriverAcc}
\paragraph{Description :} Fonction récursive du GameDriver.

\begin{itemize}
\item NewListOfQuestions
\item NewDataBase
\item BuildDecisionTreeWithQ
\item GameDriverAcc
\end{itemize}

Complexité temporelle : $\Theta(n_1 * O(BuildDecisoinTree))$

Le pire cas est lorque l'utilisateur ne connait pas la réponse, il faut donc faire appel n fois à la fonction BuildDecisionTreeWithQ

\end{document}